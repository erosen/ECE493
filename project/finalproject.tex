\documentclass[11pt]{article} % use larger type; default would be 10pt

%%% PAGE DIMENSIONS
\usepackage{geometry} % to change the page dimensions
\geometry{a4paper} 

% Margins
\topmargin=-0.45in
\evensidemargin=0in
\oddsidemargin=0in
\textwidth=6.5in
\textheight=9.0in
\headsep=0.25in

%%% PACKAGES
\usepackage{booktabs} % for much better looking tables
\usepackage{array} % for better arrays (eg matrices) in maths
\usepackage{paralist} % very flexible & customisable lists (eg. enumerate/itemize, etc.)
\usepackage{verbatim} % adds environment for commenting out blocks of text & for better verbatim


%%% HEADERS & FOOTERS
\usepackage{fancyhdr} % This should be set AFTER setting up the page geometry
\pagestyle{fancy} % options: empty , plain , fancy
\renewcommand{\headrulewidth}{0pt} % customise the layout...
\lhead{}\chead{}\rhead{}
\lfoot{}\cfoot{\thepage}\rfoot{}



%%% END Article customizations

%%% The "real" document content comes below...

\title{Hardware/Software Design of Embedded Systems \\Final Project}
\date{}

\begin{document}
\maketitle

\section{Introduction}

For the remainder of this course you will be required to work on and complete a project on a topic that interests you and your team. Before beginning, you should create an account at http://www.github.com and email your username to the TA. You should also familiarize yourself with Git and how to use version control systems. The tutorials listed on the Github website will be helpful in getting started.

\section{Requirements}
All projects must meet the following requirements.
\begin{itemize}
	\item You must work with {\emph at least} one other person with a maximum of 4 people in one group.
	\item The project size should be relative to the size of the team to ensure that each member has an appropriate amount of work.
	\item You must use at least one peripheral port from the FPGA. This includes IR, Audio, VGA, Ethernet, USB, PS2, Serial, or Composite Video. 
	\item Teams with more than two people are encouraged to come up with a project that requires an additional FPGA board but this is not required. 
	\item All work should be completed with Git version control, each team will be assigned a repository to work in. 
\end{itemize}

\section{Project Ideas}
Some possible suggestions for projects include:

\begin{itemize}
	\item Audio manipulation
	\item Video games
	\item Web server
	\item IRC chat clients
	\item IR remote control
	\item Hardware implantations of algorithms
	\item Communication protocols
	\item Other ideas can come from a simple Google search
\end{itemize}


\section{Deliverables}
\subsection{Progress Report}

Two weeks after the start of the project, you will be required to upload a one page document describing current progress and input on possible problems. Make sure to include what has been completed thus far and how you plan to complete the rest of the project on time.

\subsection{Completed Project}
The following should be uploaded to sakai by all members from your group

\begin{itemize}
	\item A powerpoint presentation describing the features of your project.
	\item A text file containing a list of all the people working on the project and their contributions. Also make sure to include a link to your Github repository.
\end{itemize}

Your entire project will be graded directly from Github. All projects will be locked at midnight on the last day of lab. Commits published after midnight will be rolled back and not accepted. 

\end{document}