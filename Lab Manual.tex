\documentclass[hidelinks]{article}
\usepackage{amsmath}
\usepackage{caption}
\usepackage{graphicx}
\usepackage{import}
\usepackage{hyperref}
\usepackage{cleveref}
\usepackage{multicol}
\usepackage[square,sort,comma,numbers]{natbib}
\usepackage[abs]{overpic}
\usepackage{setspace}
\usepackage{subcaption}
\usepackage{wrapfig}
\usepackage[upright]{fourier} 
\usepackage[usenames,dvipsnames]{xcolor}
\usepackage{tkz-euclide}
\usepackage{listings}
\usepackage{amsmath}


\usepackage{fancyhdr} % Required for custom headers
\usepackage{extramarks} % Required for headers and footers

% Margins
\topmargin=-0.45in
\evensidemargin=0in
\oddsidemargin=0in
\textwidth=6.5in
\textheight=9.0in
\headsep=0.25in

\linespread{1.1} % Line spacing

% Set up the header and footer
\pagestyle{fancy}
\lhead{EMBEDDED SYSTEMS LAB } % Top left header
\chead{} % Top center head
\rhead{14:332:435 / 16:332:519:ST} % Top right header
\lfoot{Developed by Elie Rosen} % Bottom left footer
\cfoot{} % Bottom center footer
\rfoot{Page\ \thepage} % Bottom right footer
\renewcommand\headrulewidth{0.4pt} % Size of the header rule
\renewcommand\footrulewidth{0.4pt} % Size of the footer rule

\setlength\parindent{0pt} % Removes all indentation from paragraphs

\usetkzobj{all} 
\usepackage{float}
\restylefloat{table}

\renewcommand*\thesection{\arabic{section}}


%----------------------------------------------------------------------------------------
%	CODE INCLUSION CONFIGURATION
%----------------------------------------------------------------------------------------


\definecolor{MyDarkGreen}{rgb}{0.0,0.4,0.0} % This is the color used for comments
\lstloadlanguages{VHDL} % Load Perl syntax for listings, for a list of other languages supported see: ftp://ftp.tex.ac.uk/tex-archive/macros/latex/contrib/listings/listings.pdf
\lstset{language=VHDL, % Use Perl in this example
        frame=single, % Single frame around code
        basicstyle=\small\ttfamily, % Use small true type font
        keywordstyle=[1]\color{Blue}\bf, % Perl functions bold and blue
        keywordstyle=[2]\color{Purple}, % Perl function arguments purple
        keywordstyle=[3]\color{Blue}\underbar, % Custom functions underlined and blue
        identifierstyle=, % Nothing special about identifiers                                         
        commentstyle=\usefont{T1}{pcr}{m}{sl}\color{MyDarkGreen}\small, % Comments small dark green courier font
        stringstyle=\color{Purple}, % Strings are purple
        showstringspaces=false, % Don't put marks in string spaces
        tabsize=5, % 5 spaces per tab
        morekeywords=,
        morekeywords=[2]{on, off, interp},
        morekeywords=[3],
        morecomment=[l][\color{Blue}]{...}, % Line continuation (...) like blue comment
        numbers=left, % Line numbers on left
        firstnumber=1, % Line numbers start with line 1
        numberstyle=\tiny\color{Blue}, % Line numbers are blue and small
        stepnumber=1 % Line numbers go in steps of 5
}

\newcommand{\vhdlcode}[2]{
\begin{itemize}
\item[]\lstinputlisting[caption=#2,label=#1]{#1.vhd}
\end{itemize}
}

\begin{document}

%%%% Cover Page %%%%%
\begin{titlepage}

\newcommand{\HRule}{\rule{\linewidth}{0.5mm}} % Defines a new command for the horizontal lines, change thickness here

\center % Center everything on the page

\includegraphics{RU_INF_SEAL_CMYK}\\[1cm] % Include a department/university logo - this will require the graphicx package

\textsc{\LARGE Rutgers, the State University of New
Jersey}\\[1.5cm] % Name of your university/college
\textsc{\Large ECE435/519 Special Topics}\\[0.5cm] % Major heading such as course name

\HRule \\[0.4cm]
{ \huge \bfseries Hardware/Software Design of Embedded Systems Laboratory}\\[0.4cm] % Title of your document
\HRule \\[1.5cm]

{\large Fall 2014}\\[3cm] 

Last Updated:

\today
\vfill % Fill the rest of the page with whitespace

\end{titlepage}

%%%% ToC %%%%
\tableofcontents
\newpage

%%%% Lab Content %%%%%

\section{Lab 1 - Introduction to FPGA's and VHDL}

\subsection{Introduction}
This lab will introduce you to the Altera DE2-115 FPGA Development Board. The DE2-115 contains all of the hardware necessary to prototype and create various hardware configurations on the Altera Cyclone IV FPGA chip that will be used throughout the course of this lab. By completeing this lab, you will have an understanding of all the hardware contained on the FPGA development board, along with an understanding of how to connect peripherals to the development board. Lastly, this lab will go over the standard template for designing hardware in the VHDL programming language. All this will be accomplished by following the Quartus II introductory packet along with the following activities.

\subsection{VHDL Basics}

The following code block shows how to interact with the switches and LEDs on the DE2-115. Notice how the program begins with importing the ieee library which contains all of the basic logic primitives as established within the IEEE standard 1164. When working in industry it is common for large companies to create their own libraries as well. Every VHDL file should contain at least one entity (module) that is the same as the name of the file. An entity contains information about the structure of the module such as how many inputs/outputs (I/O) and what type of logic to expect at the I/O. Finally we define the entity in an architecture block, this section does the work on the hardware. As can be seen, this code is setting the red LEDs as defined in the array to the accompanying switches on the board. Take note on the use of comments throughout the code, comments begin with two dashes (--) and should always be used to describe what you are trying to accomplish, this way someone else who reads your code will understand it easily and your code will look more professional. 

\begin{lstlisting}
-- Import logic primitives
LIBRARY ieee;
USE ieee.std_logic_1164.all;

-- Simple module that connects the SW switches to the LEDR lights
ENTITY lab1 IS
PORT ( SW: IN STD_LOGIC_VECTOR(17 DOWNTO 0); -- Initialize switches as an input
	LEDR: OUT STD_LOGIC_VECTOR(17 DOWNTO 0)); -- Initialize red LEDs as an output
END lab1;

-- Define characteristics of the entity lab1
ARCHITECTURE Behavior OF lab1 IS
BEGIN
	LEDR <= SW; -- Assign each switch to one red LED
END Behavior;
\end{lstlisting}


\subsection{Activities}

\subsubsection{Implementing Logic}

Implement the hardware from the circuit in Figure \ref{fig:circuit1}. The inputs should come from SW(1) and SW(2) and the output should be shown on any of the available LEDs. Use the implemented circuit to test and create a truth table with your results and place it within a comment in the program file.

\begin{figure}[H]
	\centering
	\includegraphics[width=100mm]{Lab1/figures/circuit1.png}
	\caption{Circuit for activity 1}
	\label{fig:circuit1}
\end{figure}

\subsubsection{7 Segment Display Decoder}

The 7-segment display is comprised of 7 LEDs that are arranged in such a way that allows for the creation of the numbers 0-9 and a select few characters with some clever use. Figure \ref{fig:7seg} shows the block diagram and output table. Your task is to create a 3 input, 7 output decoder that will display a number from 0-6. To accomplish this task, you should program the switches SW(0) - SW(6) to make the first 7 displays show the numbers 0-6 when its switch is turned on.

\begin{figure}[H]
	\centering
	\includegraphics[width=100mm]{Lab1/figures/7seg.png}
	\caption{7 segment display and decoder}
	\label{fig:7seg}
\end{figure}

{\bf Tips:} 
\begin{itemize}
  \item The eight 7 segment displays can be accesed with the 7-bit signal vectors HEX0\ldots HEX7. For example, to output to the first display (HEX0) you can either set each bit individually (HEX0(5) <= `1';) or set the whole vector with (HEX0 <= `1111111') which would display the number 8. 
  \item The second item
  \item The third etc \ldots
\end{itemize}



\newpage

\section{Lab 2 - Latches, Flip-Flops, and Counters}

\subsection{Introduction}
Elementary latches and flip-flops have been used for years as a means to store temporary data either from the outputs of logic operations or by setting them to configure logic to behave in certain ways. This lab will go into the aspects of creating latches and flip-flops which will then be used to create a counter. 

\subsection{Pre-lab}
Before you begin this lab you should complete the following and upload to Sakai:

\begin{itemize}
	\item Write down the truth table for a D-latch, SR-latch and J-K flip-flop
	\item Design a block diagram for an 8-bit synchronous counter using J-K flip-flops
\end{itemize} 

\subsection{Lab Activities}

\subsubsection{Latches}
The following VHDL code implements the logic for a D-latch based off of the schematic in Figure \ref{fig:dlatch}. 

\begin{lstlisting}
-- A gated D latch
LIBRARY ieee;
USE ieee.std_logic_1164.all;

ENTITY dlatch IS
	PORT (	CLK, D	: IN	STD_LOGIC;
		Q, Qbar	: OUT	STD_LOGIC);
END dlatch;

ARCHITECTURE rtl OF dlatch IS

	SIGNAL D1, D2, Qa, Qb : STD_LOGIC; -- Intermediate signals
	ATTRIBUTE keep: boolean; -- For waveform results
	ATTRIBUTE keep of D1, D2, Qa, Qb : signal is true;
	
BEGIN
	
	D1 <= NOT (D AND CLK);
	D2 <= NOT (D1 AND CLK);
	Qa <= NOT (D1 AND Qb);
	Qb <= NOT (D2 AND Qa);

	Q <= Qa;
	Qbar <=Qb;
	
END rtl;
\end{lstlisting}


\begin{figure}[H]
	\centering
	\includegraphics[width=100mm]{Lab2/figures/dlatch.png}
	\caption{D-latch circuit and block diagram}
	\label{fig:dlatch}
\end{figure}

Using this as a reference, design VHDL for an SR-latch with a clock input. Verify with waveforms that the circuit behaves the same as the truth table you created in the pre-lab.


\subsubsection{Flip-Flop}
Design VHDL code that implements the logic for a J-K flip-flop from Figure \ref{fig:jkflipflop}. Verify with waveforms that the circuit behaves the same as the truth table you created in the pre-lab 

\begin{figure}[H]
	\centering
	\includegraphics[width=100mm]{Lab2/figures/jkflipflop.png}
	\caption{J-K fip-flop circuit}
	\label{fig:jkflipflop}
\end{figure}

\subsubsection{Counters}
In the pre-lab, you created a block diagram for an 8-bit counter using J-K flip flops. Using the same VHDL code you created for implementing the J-K flip-flop, implement an 8-bit counter that increments when you press KEY0 on the DE2-115 development board. Link the binary output of the flip-flops to the red LEDs and then convert the binary value into hexadecimal to be shown on the 7-segment displays. \emph{Hint: you should refer to your code for the 7-segment display driver designed in the previous lab.}

\subsection{Lab Report}
Your lab report submission should break down as follows:
\begin{itemize}
	\item Extract from the fpga lab folder the VHDL file from each project, upload each of them to the Sakai Assignment page. Ex: part1.vhdl and part2.vhdl
	\begin{itemize}
		\item Make sure that your code is well commented.
	\end{itemize}
	\item Waveforms for the SR latch and JK flipflop (if JK doesn't work, try implementing it as a truth table with if statements)
	\item For Part 3 after compilation go to Tools > Netlist Viewers > RTL Viewer. Print a pdf of this page and discuss what you see.
	\item In a separate txt or pdf document, prepare a discussion on your compilation results. Be sure to include specifics about the amount of hardware used and where the numbers in the compilation results came from. Do the numbers make sense? Why? How do these results impact the use of FPGAs in industry? This should be no more than one page long, less is preferred.
\end{itemize}


\newpage

\section{Lab 3 - Complex Addition Systems}

\subsection{Introduction}
From your pocket calculator to inside modern CPUs adders have long been used as more than just a simple way to sum numbers together. This lab will go into the logic structure of the adder as well as provide a method for converting the simple full adder in to an \emph{arithmetic logic unit} (ALU) capable of handling 12 operations.

\subsection{Pre-lab}
Before coming to the lab, please complete the following and upload to Sakai:
\begin{itemize}
	\item A truth table for a half adder and a full adder
	\item The logic equation for a 4-bit ripple carry adder
\end{itemize}

\subsection{Lab Activities}

\subsubsection{Half Adder}
Build the circuit in Figure \ref{fig:halfadder}, create waveforms to verify that the logic is correct.

\begin{figure}[H]
	\centering
	\includegraphics[width=60mm]{Lab3/figures/halfadder.png}
	\caption{Circuit for a 1-bit half adder}
	\label{fig:halfadder}
\end{figure}

\subsubsection{Full Adder}
The circuit in Figure \ref{fig:fulladder} implements a full 1-bit adder. Implement this circuit in VHDL, create a waveform, and verify that the logic behaves as expected. 

\begin{figure}[H]
	\centering
	\includegraphics[width=80mm]{Lab3/figures/fulladder.png}
	\caption{Circuit for a 1-bit full adder}
	\label{fig:fulladder}
\end{figure}

Now that you have a working 1-bit full adder, implement a 4-bit ripple carry adder that sums the binary numbers "0110" and "0101." A block diagram for the 4-bit ripple carry adder is shown in Figure \ref{fig:fourbitripple}. Verify your results by creating a waveform and simulating the circuit.

\begin{figure}[H]
	\centering
	\includegraphics[width=100mm]{Lab3/figures/fourbitripple.png}
	\caption{4-bit ripple carry adder block diagram}
	\label{fig:fourbitripple}
\end{figure}

\subsubsection{Full Adder Based ALU}
The block diagram in Figure \ref{fig:fulladderalu} is an example of how the regular 1-bit full adder can be manipulated to implement additional functionality. For this activity, you must build VHDL code that implements a 4-bit complex adder ALU, the list of instructions can be found in Table \ref{tab:adderaluop}.

\begin{figure}[H]
	\centering
	\includegraphics[width=150mm]{Lab3/figures/fulladderalu.png}
	\caption{Block diagram for a 4-bit ripple carry adder ALU with 12 operations}
	\label{fig:fulladderalu}
\end{figure}

After writing and testing your VHDL code, upload it on the DE2-115 FPGA Development board. Connect the inputs {\bf A3-A0} to SW3 - SW0 and the inputs {\bf B3-B0} to SW4(7)- SW(4). Connect the select lines for the {\bf A} multiplexer to SW(8) and SW(9) while the select lines for {\bf B} should connect to SW10 and SW11. Lastly, the carry in input should connect to SW12. Display inputs {\bf A} and {\bf B} in hexadecimal on HEX7 and HEX5, respectively and the output {\bf S}, on HEX3. If the result over-flows display {\bf C$_4$} on LEDG0. If the result is zero turn on LEDR1. Test and verify all twelve operations are correct and make a table that includes the values for {\bf A}, {\bf B}, {\bf S}, {\bf C$_4$}, and {\bf Zero} for each operation.

\begin{table}[H]
	\centering
	\caption{List of opperations for the adder based ALU}
	\begin{tabular}{ | c | c | c | c | }
		\hline                        
 		\bf A$_i$ & \bf B$_i$ & \bf Carry In & \bf Result \\ \hline
 		Set to 0 & Set to 0 & 0 & 0 \\ \hline
 		Set to 0 &  Set to 0 & 1 & 1 \\ \hline
 		A &  Set to 0 & 0 & A \\ \hline
 		Set to 0 & B  & 0  & B  \\ \hline
 		A & Set to 0 & 1 & A $+$ 1 \\ \hline
 		Set to 0 & B & 1 & B $+$ 1 \\ \hline
 		A & B & 0 & A $+$ B \\ \hline
 		Set to invert & B & 1 & B $-$ A \\ \hline
 		Set to invert & Set to 0 & 0 & $\overline{A}$ \\ \hline
 		Set to invert & Set to 0 & 1 & $-$A \\  \hline
		Set to 0 & Set to invert & 0 & $\overline{B}$ \\ \hline
 		Set to 0 & Set to invert & 1 & $-$B \\ 
 		\hline
	\end{tabular}
	\label{tab:adderaluop}
\end{table}

When you complete building the VHDL upload your code to the FPGA board. Test and verify all twelve operations are correct.

\subsection{Lab Report}
Your lab report submission should break down as follows:
\begin{itemize}
	\item Extract from the fpga lab folder the VHDL file from each project, upload each of them to the Sakai Assignment page. Ex: part1.vhdl and part2.vhdl
	\begin{itemize}
		\item Make sure that your code is well commented.
	\end{itemize}
	\item Waveforms for the Half Adder and Full Adder.
	\item For Part 3 after compilation go to Tools > Netlist Viewers > RTL Viewer. Print a pdf of this page and discuss what you see.
	\item Your report should include a one page discussion on the number of pins used and your table from part 3.
\end{itemize}


\newpage

\section{Lab 4 - Finite-State Machines}

\subsection{Introduction}
A \emph{finite-state machine} (FSM) is a system that can be in only one state at a time; the state it is in at any given time is called the current state. It can change from one state to another when initiated by a triggering event or condition; this is called a transition. A particular FSM is defined by a list of its states, and the triggering condition for each transition. This lab will examine a simple FSM to give the general idea of how they work and build upon it to create a system modeled off of a real world system, a vending machine.

\subsection{Pre-lab}

Before coming to the lab, please complete the following and upload to Sakai:
\begin{itemize}
	\item By examining the code in Listing \ref{code:fsm} write out its state machine transition graph.
\end{itemize}


\subsection{Lab Activities}

\subsubsection{FSM}

Given the state machine transition graph in Figure \ref{fig:fsm}, design VHDL for the state transitions of this diagram. Note that in the state transition graph, we show the transition behavior as D0 D1 = 00 / 1 0 to indicate that the transition occurs when D0 = 0 and D1 = 0, and causes Z1 to be 1 and Z2 to be 0. Assume that state A is the reset state for this machine.

\begin{figure}[H]
	\centering
	\includegraphics[width=100mm]{Lab4/figures/fsm.png}
	\caption{State machine transition graph}
	\label{fig:fsm}
\end{figure}

You may find the code below to be helpful in completing this activity. 

\begin{lstlisting}[caption=Sample code for a Finite State Machine, label=code:fsm]
-- User-Encoded State Machine
library ieee;
use ieee.std_logic_1164.all;

entity state_machine is
	port(clk	 	 : in std_logic;
		reset	  : in std_logic;
		input	  : in std_logic;
		output	  : out std_logic);
	
end entity;

architecture rtl of state_machine is
	-- Build an enumerated type for the state machine
	type count_state is (A, B, C, D);
	
	-- Registers to hold the current state and the next state
	signal present_state, next_state	   : count_state;
	
	-- Attribute to declare a specific encoding for the states
	attribute syn_encoding				  : string;
	attribute syn_encoding of count_state : type is "11 01 10 00";
	
begin
	-- Move to the next state
	process(clk, reset)
	begin
		if reset = '1' then
			present_state <= A;
		elsif (rising_edge(clk)) then
			present_state <= next_state;
		end if;
	end process;

	-- Determine what the next state will be, and set the output bits
	process (present_state, input)
	begin
		case present_state is
			when A =>
				if (input = '0') then
					next_state <= B;
					output <= '0';
				else
					next_state <= D;
					output <= '0';
				end if;
			when B =>
				if (input = '0') then
					next_state <= C;
					output <= '1';
				else
					next_state <= A;
					output <= '0';
				end if;
			when C =>
				if (input = '0') then
					next_state <= D;
					output <= '0';
				else
					next_state <= B;
					output <= '1';
				end if;
			when D =>
				if (input = '0') then
					next_state <= A;
					output <= '0';
				else
					next_state <= C;
					output <= '1';
				end if;
		end case;
	end process;
	
end rtl;
\end{lstlisting}

\subsubsection{Vending Machine}

Design a custom finite-state machine to control a vending machine to dispense products. The design must follow the following specifications:

\begin{enumerate}

	\item The state machine should have three states:

	\begin{itemize}
		\item \emph{IDLE}: This is the default state for the machine. The machine should stay in this state until a product is selected. If no items are selected, display a dash across all HEX displays. This state should also set the 6-bit signal QUARTERS and the signal DISPENSE\_READY to zero as well as LEDR0 - LEDR17. KEY1 should set the signal COIN\_RETURN to HIGH which will reset the machine.

		\item \emph{PRODUCT\_SELECT}: The state machine will move into this state if product(s) are selected through use of the switches (SW0 - SW15) on the FPGA board. Display the total cost in number of quarters needed of all products selected on HEX5 - HEX4 in decimal.  KEY3 and KEY2 should increment QUARTERS by a dollar and quarter respectively when pressed. Display the value of QUARTERS on HEX1 - HEX0. When the correct number of quarters have been inserted, the signal DISPENSE\_READY should go HIGH and the state should transition to DISPENSE. If KEY1 is pressed, return to the IDLE state. A list of available products and their cost can be found in Table \ref{tab:costlist}.

		\item \emph{DISPENSE}: Once the proper amount of quarters have been deposited, the item should be dispensed from the machine. To show that the item(s) have been dispensed, turn on LEDR0-LEDR17 and set the next state to idle.

	\end{itemize}
	
	\item KEY0 will act as a CLOCK input. The states should transition on the rising edge. This means that you should select a product and then send a CLOCK pulse to calculate the cost of the item(s) selected. Then add quarters into the machine and send another CLOCK pulse when there is the right amount of quarters. If there is not enough quarters inserted, the state should not change. Once the item(s) are dispensed, send another CLOCK pulse to return to IDLE.

	\item The current state; \emph{IDLE}, \emph{PRODUCT\_SELECT}, and \emph{DISPENSE} should be displayed by turning on LEDG0 - LEDG2 respectively.
	
\end{enumerate}

\begin{table}[H]
	\centering
	\caption{List of vending machine items, cost, and switch correspondence}
	\begin{tabular}{ | c | c | c | }
		\hline                        
 		\bf Product & \bf Cost & \bf Switch\\ \hline
 		SODA\_CAN[0:3] & \$1.00 & SW12 - SW15 \\ \hline
		CHIPS[0:3] & \$0.75 & SW8 - SW11 \\ \hline
		CHOCOLATE[0:3] & \$0.50 & SW4 - SW7 \\ \hline
		BUBBLE\_GUM[0:3] & \$0.25 & SW0 - SW3 \\
 		\hline
	\end{tabular}
	\label{tab:costlist}
\end{table}


\subsection{Lab Report}

After completing the activities in this lab you should create a zip folder with the following and then submit it to Sakai:

\begin{itemize}
	\item Commented VHDL code.
	\item VHDL test bench for the FSM activity.
	\item Waveform for the FSM activity.
	\item Pictures of the results from the vending machine activity.
	\item A discussion on the results of compilation including longest path delay, the total number of logic elements used, and issues you encountered while performing the lab.
\end{itemize}




\newpage

\section{Lab 5 - A Simple Computer}

\subsection{Introduction}
16bit simple processor

\subsection{Lab Activities}

\subsubsection{Design an ALU}
Add, Sub, Mult, Shift, XOR, AND, NAND,

\subsubsection{Design a RAM}
512 bit memory

\subsubsection{Design the Program Counter}
simple counter

\subsubsection{Create a VGA Driver}
Display the output onto the screen

\end{document}